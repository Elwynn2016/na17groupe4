\documentclass[11pt]{report}
\usepackage[utf8]{inputenc}
\usepackage[T1]{fontenc}
 \usepackage{listings}

\usepackage{fancyhdr}
\usepackage{graphicx}
\usepackage{color}
\pagestyle{fancy}

\title{\textbf{Etudoc}}
\author{Noémie GALZIN \\
		Yueqing JIA \\
		Elwynn NEGLOKPE\\
		Océane DUBOIS}
\date{}
\begin{document}

\maketitle
\newpage
\section{Présentation}
La plate-forme Etudoc a pour objectif de centraliser tous les rapports réalisés dans le cadre d'UVs à l'UTC. On peut y accéder, en passant par le CAS, en tant qu'utilisateur normal ou en tant qu'administrateur. 
En tant qu'utilisateur on peut y consulter les documents, faire de recherches, ainsi que déposer ses propres documents. 
L'administrateur peut, quant à lui, archiver un document, ajouter une licence, ajouter une catégorie et retirer un document de l'archive.

\section{Choix des technologies}
Pour implémenter cette plate-forme nous avons choisi d'utiliser une base de données non-relationelles que nous avons implémentée sous Oracle. 

Pour l'interface utilisateur nous avons choisi de mettre en place un programme python en ligne de commande car le python n'est pas difficile à apprendre et son interfacement avec la base de données est relativement simple.

Nous avons décidé d'utiliser python 2 car étant plus ancien, il est plus simple de trouver de la documentation et des tutoriels dessus ; des membres du groupe, personne n'avait codé en python au préalable. De plus python 2 était déjà installé par défaut sur nos ordinateurs.

\section{Fonctions implémentées}

Comme dit précédement, on peut accéder au programme en tant qu'utilisateur ou en tant qu'administrateur.

La redirection se fera à partir du CAS. Cependant, notre application n'y étant pas reliée, le choix d'accéder à la plateforme en tant qu'utilisateur ou administrateur se fait au démarrage du programme python.

\subsection{Fonctions de l'administrateur}

L'administrateur peut être un professeur ou un membre de l'administration. Ils peuvent gérer l'archive d'un document, et ajouter de nouvelles licences et catégories.

\subsubsection{Archiver un document}

L'administrateur peut donc archiver un document. Pour cela, on affiche tous les documents non archivés et l'administrateur n'a plus qu'à choisir l'identifiant du document qu'il souhaite archiver. On met à jour alors la ligne correspondante de la base de données avec la valeur d'archivedoc à 'Y'.

\subsubsection{Ajouter une licence}

Pour ajouter une licence, on commence d'abord par afficher toutes les licences déjà existantes afin que l'administrateur sache quelles licences sont déjà dans la base de données. Puis il rentre le code de la licence et sa description. Si, lors de l'insertion, on se rend compte que la licence existe déjà dans la base de données, on affiche le message d'erreur suivant : "L'élément existe déjà dans la base de données".

\subsubsection{Ajouter une Categorie}

Pour ajouter une catégorie, on suit le même principe que pour l'ajout d'une licence. On commence par afficher toutes les catégories puis l'administrateur rentre la nouvelle catégorie souhaitée. Si elle existe déjà, un message d'erreur s'affiche. Sinon, elle est insérée dans la base de données.

\subsubsection{Retirer un document de l'archive}

Pour retirer un document de l'archive on procède de la même façon que pour en archiver un. On affiche tous les documents qui sont archivés, puis l'administrateur rentre l'identifiant du document qu'il souhaite retirer de l'archive. On met à jour alors la ligne correspondante de la base de données avec la valeur d'archivedoc à 'N'.

\subsection{Fonction de l'utilisateur}

L'utilisateur peut être un élève, un professeur ou même un membre de l'administration. Il peut déposer des documents et consulter tous les documents non archivés sur la plate-forme.

\subsubsection{Ajouter un document}

Lors de l'ajout d'un document, l'utilisateur rentre successivement tous les champs nécessaires à l'ajout d'un document. L'identifiant unique du document est généré de cette manière : on sélectionne l'identifiant maximum déjà présent dans la base de données et on lui ajoute 1. 
On peut désigner plusieurs élèves et plusieurs professeurs comme auteurs et suiveurs du projet.

\subsubsection{Voir toutes les informations d'un document}

Cette fonction permet d'afficher toutes les informations d'un document. 
On commence par sélectionner l'identifiant du document souhaité puis le programme affiche toutes les informations le concernant présentes dans la base de données.

\subsubsection{Rechercher un document}
On peut donc rechercher un document. Tout d'abord nous avons implémenté les fonctions de recherche uniquement par catégorie, uniquement par mot clé, uniquement par semestre ou uniquement par auteur. Puis nous avons implémenté une fonction de recherche qui permet de renvoyer le resultat d'une recherche effectuée avec plusieurs critères : une catégorie, un mot clé et un semestre par exemple. Cette fonction fait l'union des 3 requêtes et renvoie donc les documents qui correspondent à chaque champ recherché.

\begin{itemize}
\item Recherche par catégorie : La recherche par catégorie affiche toutes les catégories disponibles, puis l'utilisateur entre la catégorie choisie. Les documents de cette catégorie sont ensuite affichés. On peut alors faire appel à la fonction permettant de voir toutes les informations d'un document à partir de son identifiant. 

\item Recherche par mots-clés : La recherche par mots-clés affiche tous les mots-clés disponibles, puis l'utilisateur entre le mot-clé choisi. Les documents correspondants sont alors affichés. On peut ensuite faire appel à la fonction permettant de voir toutes les informations d'un document à partir de son identifiant. 

\item Recherche par semestre : Dans la recherche par semestre on demande à l'utilisateur de rentrer une saison (P ou A). Ensuite, on vérifie que la saison rentrée est correcte. Puis on lui demande de rentrer une année, et on vérifie que celle-ci est comprise entre 1972 et l'année en cours. Si c'est le cas, on affiche tous les documents correspondant à la requête. On peut ensuite faire appel à la fonction permettant de voir toutes les informations d'un document à partir de son identifiant.  

\item Recherche par auteur : Dans la recherche par auteur, on propose d'abord de renter le nom de l'auteur souhaité. De là, on affiche tous les étudiant ayant ce nom, ce qui nous permet d'avoir le login de l'étudiant souhaité. Grâce à lui, on affiche tous les document pour lesquels l'un des auteurs est l'élève ayant ce login.

\item Recherhe par catégorie ou par mot-clé ou par semestre : Comme expliqué précédement, dans cette fonction on peut choisir d'entrer plusieurs données (catégorie, mot-clé, semestre) ; la fonction retourne les documents correspondant à au moins une des données rentrées.
\end{itemize}

\end{document}

